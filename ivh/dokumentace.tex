  \documentclass[11pt,a4paper]{article}
% rozmery stranky
\usepackage[left=1.5cm,text={18cm, 25cm},top=2.5cm]{geometry}
% cestina a fonty
\usepackage[czech]{babel}
\usepackage[utf8]{inputenc}
\usepackage[T1]{fontenc}
% dalsi balicky
\usepackage{graphicx}
\usepackage{enumitem}
\usepackage{indentfirst}
\usepackage{url}
\usepackage[bookmarksopen,colorlinks,plainpages=false,urlcolor=blue,
            unicode,linkcolor=black]{hyperref}

\begin{document}

  \begin{titlepage}
    \begin{center}
      \Huge
      \textsc{Fakulta informačních technologií\\ Vysoké učení technické v~Brně}
      \vspace{100px}
      \begin{figure}[!h]
        \centering
        \includegraphics[height=5cm]{logo.jpg}
      \end{figure}
      \\[50mm]
      \LARGE{Dokumentace k~projektu předmětu IVH \,--\, Seminář VHDL
             \newline
             Hodiny a stopky}
      \vfill
    \end{center}
    \Large{Roman Blanco (xblanc01@stud.fit.vutbr.cz) \hfill 1.5.2015}
  \end{titlepage}

  \tableofcontents
  \newpage

  \section{Úvod}
 
    Tento dokument má za cíl popsat řešení k projektu do předmětu Semimář
    VHDL \,--\, implementace hodin a stopek v jazyce pro popis
    hardwaru \,--\, VHDL.

  \section{Zadání}
    
    Zadáním projektu bylo implementovat jednoduché hodiny (ve formátu HH:MM:SS) a
    stopky ke školní pomůcce FITkit, které budou zobrazováný na monitoru,
    připojeném k FITkitu pomocí VGA rozhraní.

Pro ovládání programu při práci se stopkami bude využita klávesnice, která je součástí FITkitu.
    Implementovány mají být funkce pro spuštění stopek, zastavení a
    vynulování.
    Pro usnadnění práce je v projektu povoleno využít mikrokontroler k
    obsluze klávesnice a nastavení hodin, samotná logika hodin však musí být
    implementována v FPGA.

  \section{Návrh řešení}

    Zobrazení hodin a stopek na VGA rozhraní bude probíhat v rozlišení
    $640\times480$. Jednotlivá čísla budou uloženy v paměti ROM, přičemž
    paměťový nárok na jednu číslici je $6\times8$ bitů.

    Počítání času bude realizováno generováním signálu s periodou 1 sekunda,
    což umožní postupně inkrementovat hodnoty signálu reprezentující časový
    údaj.

    Číselná hodnota hodin bude uložena do 6 registrů, přičemž hodnoty v těchto
    registrech mohou nabývat hodnoty 0 až 9. Každý registr bude představovat
    jednu číslici v časovém údaji.

    Hodnota těchto registrů slouží jako index k patřičnému číslu v paměti ROM.

    Pro funkčnost stopek bude využit Moorův konečný automat, který bude pracovat se
    stavy: \texttt{START}, \texttt{STOP} a \texttt{CLEAR}.

    Popis stavů:

    \begin{itemize}
      \item \texttt{START} spuštění stopek
      \item \texttt{STOP} zastavení stopek
      \item \texttt{CLEAR} vynulování stopek
    \end{itemize}

\end{document}
