\documentclass[11pt,a4paper,twocolumn]{article}
% rozmery stranky
\usepackage[left=1.5cm,text={18cm, 25cm},top=2.5cm]{geometry}
% cestina a fonty
\usepackage[czech]{babel}
\usepackage[utf8]{inputenc}
\usepackage[T1]{fontenc}
% dalsi balicky
\usepackage{times}
\usepackage{amsfonts}
\usepackage{amsthm}
\usepackage{amsmath}

\theoremstyle{plain}
\newtheorem{theorem}{Algoritmus}[section]

\theoremstyle{definition}
\newtheorem{bg}[theorem]{Definice}

\theoremstyle{plain}
\newtheorem{pov}{Věta}

\begin{document}

  \begin{titlepage}
    \begin{center}
      \Huge
      \textsc{Fakulta informačních technologií \\ Vysoké učení technické v~Brně}
      \\[84mm]
      \LARGE{Typografie a publikování -- 2. projekt \\
             Sazba dokumentů a matematickýh výrazů}
      \vfill
    \end{center}
    \Large{2014 \hfill Roman Blanco}
  \end{titlepage}

  \section*{Úvod}

    V~této úloze si vyzkoušíme sazbu titulní strany, matematických vzorců, 
    prostředí a dalších textových struktur obvyklých pro technicky zaměřené 
    texty (například rovnice (\ref{rce}) nebo definice 
    \ref{bezkontextovaGramatika} na straně \pageref{bezkontextovaGramatika}).

    Na titulní straně je využito sázení nadpisu podle optického středu
    s~využitím zlatého řezu. Tento postup byl probírán na přednášce.

  \section{Matematický text}

    Nejprve se podíváme na sázení matematických symbolů a výrazů v~plynulém
    textu. Pro množinu $V$ označuje $\mathrm{card}(V)$ kardinalitu $V$. 
    Pro množinu $V$ reprezentuje $V^{*}$ volný monoid generovaný množinou $V$
    s~operací konkatenace.
    Prvek identity ve volném monoidu $V^{*}$ značíme symbolem $\varepsilon$.
    Nechť $V^{+}=V^{*}-\{\varepsilon\}$. Algebraicky je tedy $V^{+}$ volná 
    pologrupa generovaná množinou $V$ s~operací konkatenace.
    Konečnou neprázdnou množinu $V$ nazvěme \emph{abeceda}.
    Pro $w \in V^{*}$ označuje $|w|$ délku řetězce $w$. Pro $W \subseteq V$
    označuje $\mathrm{occur}(w,W)$ počet výskytů symbolů z~$W$ v~řetězci $w$ a
    $\mathrm{sym}(w,i)$ určuje $i$-tý symbol řetězce $w$; například
    $\mathrm{sym}(abcd,3) = c$.

    Nyní zkusíme sazbu definic a vět s~využitím balíku \texttt{amsthm}.

    \begin{bg} \label{bezkontextovaGramatika}
      \emph{Bezkontextová gramatika} je čtveřice $G=(V,T,P,S)$, kde $V$ je
      totální abeceda,
      $T \subseteq V$ je abeceda terminálů, $S \in (V-T)$ je startující symbol
      a $P$ je konečná množina \emph{pravidel} tvaru $q:A\rightarrow\alpha$,
      kde $A \in (V-T)$, $\alpha \in V^{*}$ a $q$ je návěští tohoto pravidla. 
      Nechť $N=V-T$ značí abecedu neterminálů. Pokud $q: A \rightarrow \alpha
      \in P$, $\gamma$, $\delta \in V^{*}$, $G$ provádí derivační krok z~$\gamma
      A \delta$ do $\gamma \alpha \delta$ podle pravidla $q:A \rightarrow
      \alpha$, symbolicky píšeme $\gamma A \delta \Rightarrow \gamma \alpha
      \delta [q:A \rightarrow \alpha]$ nebo zjednodušeně $\gamma A \delta
      \Rightarrow \gamma \alpha \delta$. Standardním způsobem definujeme
      $\Rightarrow^{m}$, kde $m \geq 0$. Dále definujeme tranzitivní uzávěr
      $\Rightarrow^{+}$ a tranzitivně-reflexivní uzávěr $\Rightarrow^{*}$.
    \end{bg}

    Algoritmus můžeme uvádět podobně jako definice textově, nebo využít
    pseudokódu vysázeného ve vhodném prostředí (například
    \texttt{algorithm2e}).

    \begin{theorem}
      Algoritmus pro ověření bezkontextovosti gramatiky. Mějme gramatiku 
      $G = (N, T, P, S)$.
      \begin{enumerate}
        \item \label{prvni}Pro každé pravidlo $p \in P$ proveď test, zda $p$ na
          levé straně obsahuje právě jeden symbol z~$N$.
        \item Pokud všechna pravidla splňují podmínku z~kroku \ref{prvni}, 
        tak je gramatika $G$ bezkontextová.
      \end{enumerate}
    \end{theorem}

    \begin{bg}
      Definice: Jazyk definovaný gramatikou $G$ definujeme jako
      $L(G)=\{w \in T^{*}|S \Rightarrow^{*}w\}$.
    \end{bg}

    \subsection{Podsekce obsahující větu}

      \begin{bg}
        Nechť $L$ je libovolný jazyk. $L$ je \emph{bezkontextový jazyk}, když a
        jen když $L=L(G)$, kde $G$ je libovolná bezkontextová gramatika.
      \end{bg}

      \begin{bg}
        Množinu $\mathcal{L}_{CF} = \{L|L$ je bezkontextový jazyk $\}$ nazýváme
        \emph{třídou bezkontextových jazyků}.
      \end{bg}

      \begin{pov} \label{vetajedna}
        Nechť $L_{abc} = \{a^{n}b^{n}c^{n}|n \geq 0 \}$. Platí, že $L_{abc}
        \notin \mathcal{L}_{CF}$.
      \end{pov}

      \begin{proof}
        Důkaz se provede pomocí Pumping lemma pro bezkontextové jazyky, kdy
        ukážeme, že není možné, aby platilo, což bude implikovat pravdivost
        věty \ref{vetajedna}.
      \end{proof}

  \section{Rovnice a odkazy}

    Složitější matematické formulace sázíme mimo plynulý text. Lze umístit
    několik výrazů na jeden řádek, ale pak je třeba tyto vhodně oddělit,
    například příkazem $\backslash$ \texttt{quad}. 

    $$ \sqrt[x^{2}]{y^{3}_{0}} \quad \mathbb{N} = \{0,1,2,...\} \quad x^{y^{y}}
    \neq x^{yy} \quad z_{i_{j}} \not \equiv z_{ij} $$

    V~rovnici (\ref{rce}) jsou využity tři typy závorek s~různou explicitně
    definovanou velikostí.
    
    \begin{eqnarray} \label{rce}
      \bigg\{\Big[(a+b)*c\Big]^{d}+1\bigg\} & = & x \\
      \lim_{x \to \infty}\frac{\sin^{2}x+\cos^{2}x}{4} &= & y \nonumber
    \end{eqnarray}

    V~této větě vidíme, jak vypadá implicitní vysázení limity
    $\lim_{n\rightarrow\infty} f(n)$ v~normálním odstavci textu. Podobně je to i
    s~dalšími symboly jako $\sum^{n}_{1}$ či $\bigcup_{A\in \mathcal{B}}$ .
    V~případě vzorce $\lim\limits_{x \to 0}\frac{\sin x}{x}=1$ jsme si vynutili
    méně úspornou sazbu příkazem $\backslash$ \texttt{limits}.

    \setcounter{equation}{1}

    \begin{eqnarray}
      \int\limits^{b}_{a} f(x)\mathrm{d}x & = & - \int^{a}_{b}f(x)\mathrm{d}x \\
      (\sqrt[5]{x^{4}})' = (x^{\frac{4}{5}})' & = & \frac{4}{5}x^{-\frac{1}{5}} 
      = \frac{4}{5\sqrt[5]{x}} \\
      \overline{\overline{A \vee B}} & = & \overline{\overline{A} \vee
      \overline{B}}
    \end{eqnarray}

  \section{Matice}

    Pro sázení matic se velmi často používá prostředí \texttt{array} a závorky
    ($\backslash$ \texttt{left}, $\backslash$ \texttt{right}). 
    \newpage

    \begin{tabular}{l r l}

      & & $
      \begin{pmatrix}
        a+b                  & b-a                     \\
        \widehat{\xi+\omega} & \hat{\pi}               \\
        \overrightarrow{a}   & \overleftrightarrow{AC} \\
        0                    & \beta
      \end{pmatrix}$ \\[11mm]

      & A~= & $
      \begin{Vmatrix}
        a_{11} & a_{12} & \cdots & a_{1n} \\
        a_{21} & a_{22} & \cdots & a_{2n} \\
        \vdots & \vdots & \ddots & \vdots \\
        a_{m1} & a_{m2} & \cdots & a_{mn}
      \end{Vmatrix}$ \\[11mm]
       
      & & $
      \begin{vmatrix}
        t & u \\
        v & w         
      \end{vmatrix} = tw - uv$\\[5mm]

    \end{tabular}

    Prostředí \texttt{array} lze úspěšně využít i jinde.

    $$ \binom{n}{k} = \left\{
    \begin{array}{l l}
      \frac{n!}{k!(n-k)!} & \text{pro } 0 \leq k \leq n \\
      0 & \text{pro } k < 0 \text{ nebo } k > n
    \end{array} \right. $$

  \section{Závěrem}

    V~případě, že budete potřebovat vyjádřit matematickou konstrukci nebo symbol
    a nebude se Vám dařit jej nalézt v~samotném \LaTeX u, doporučuji prostudovat
    možnosti balíku maker \AmS-\LaTeX.
    Analogická poučka platí obecně pro jakoukoli konstrukci v~\TeX u.

\end{document}
