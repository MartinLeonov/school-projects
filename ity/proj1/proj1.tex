\documentclass[11pt,a4paper,twocolumn]{article}
% rozměry stránky
\usepackage[left=2cm,text={17cm, 24cm},top=2.5cm]{geometry}
% čeština a fonty
\usepackage[czech]{babel}
\usepackage[utf8]{inputenc}
\usepackage[T1]{fontenc}
\usepackage[normalem]{ulem}
\newcommand{\myuv}[1]{\quotedblbase #1\textquotedblleft}
\newenvironment{centerblock}[1][\textwidth]
{
   \begin{center}
   \begin{minipage}[t]{#1}
   \raggedright
}{
   \end{minipage}
   \end{center}
}

\begin{document}

   \title{
       Typografie a publikování \\[1mm]
       1. projekt \\[2mm]}
   \author{
       Roman Blanco \\[0.5mm]
       xblanc01@stud.fit.vutbr.cz}
   \date{}

   \maketitle

   \section{Hladká sazba}

      Hladká sazba je sazba z~jednoho stupně, druhu a~řezu písma sázená na stanovenou šířku odstavce. Skládá se~z odstavců, které obvykle začínají za\-ráž\-kou, ale mohou být sázeny i~bez zarážky -- rozhodující je celková grafická úprava. Odstavce jsou ukončeny východovou řádkou. Věty nesmějí začínat číslicí. 

      Barevné zvýraznění, podtrhávání slov či různé velikosti písma vybraných slov se zde také ne\-po\-u\-ží\-vá. Hladká sazba je určena především pro delší texty, jako je například beletrie. Porušení konzistence sazby působí v~textu rušivě a~unavuje čtenářův zrak.

   \section{Smíšená sazba}

      Smíšená sazba má o~něco volnější pravidla, jak hladká sazba. Nejčastěji se klasická hladká saz\-ba doplňuje o~další řezy písma pro zvýraznění důležitých pojmů. Existuje \myuv{pravidlo}:

      \begin{centerblock}[6.5cm]
         \textsc{Čím více druhů, řezů, velikostí, barev písma a jiných efektů použijeme, tím profesionálněji bude dokument vypadat. Čtenář tím bude vždy nadšen!}
      \end{centerblock}

      Tímto pravidlem se \uline{nikdy} nesmíte řídit. Příliš časté zvýrazňování textových elementů a změny {\huge V}{\LARGE E}{\Large L}{\large I}{\normalsize K}{\small O}{\footnotesize S}{\scriptsize T}{\tiny I}\hfill{\large písma}\hfill{\Large jsou}\hfill{\LARGE známkou} \textbf{{\huge amatérismu}} autora a působí \textit{\textbf{velmi} rušivě}. Dobře navržený dokument nemá obsahovat více než 4 řezy či druhy písma. \texttt{Dobře navržený dokument je decentní, ne chaotický.}

      Důležitým znakem správně vysázeného do\-ku\-men\-tu je konzistentní používání různých druhů zvý\-raz\-ně\-ní. To například může znamenat, že \textbf{tuč\-ný řez} písma bude vyhrazen pouze pro klí\-čo\-vá slo\-va, \textit{skloněný řez} pouze pro~doposud nez\-ná\-mé pojmy a nebude se to míchat. Sklo\-ně\-ný řez nepůsobí tak rušivě a používá se častěji. V~\LaTeX u jej sázíme raději příkazem $\backslash$\texttt{emph\{text\}} než $\backslash$\texttt{textit\{text\}}.

      Smíšená sazba se nejčastěji používá pro~sazbu vědeckých článků a technických zpráv. U~delších dokumentů vědeckého či technického charakteru je zvykem upozornit čtenáře na~význam různých typů zvýraznění v~úvodní kapitole.

   \section{České odlišnosti}

      Česká sazba se oproti okolnímu světu v~některých aspektech mírně liší. Jednou z~odlišností je sazba uvozovek. Uvozovky se v~češtině používají převážně pro~zobrazení přímé řeči. V~menší míře se používají také pro~zvýraznění přezdívek a~ironie. V~češtině se používá tento \myuv{typ uvozovek} namísto anglických \myuv{uvozovek}.

      Ve smíšené sazbě se řez uvozovek řídí řezem prvního uvozovaného slova. Pokud je uvozována celá věta, sází se koncová tečka před uvozovku, pokud se uvozuje slovo nebo část věty, sází se tečka za~uvozovku.

      Druhou odlišností je pravidlo pro sázení konců řádků. V~české sazbě by řádek neměl končit osamocenou jednopísmennou předložkou nebo spojkou (spojkou \myuv{a} končit může při~sazbě do 25 liter). Abychom \LaTeX u zabránili v~sázení osamocených předložek, vkládáme mezi předložku a~slovo nezlomitelnou mezeru znakem\ \~\ (vlnka, tilda). Pro~automatické doplnění vlnek slouží volně šiřitelný program \textit{vlna} od~pana Olšáka \footnote{Viz ftp://math.feld.cvut.cz/pub/olsak/vlna/.}.


\end{document}
