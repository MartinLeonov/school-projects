\documentclass[11pt,a4paper]{article}
% rozmery stranky
\usepackage[left=1.5cm,text={18cm, 25cm},top=2.5cm]{geometry}
% cestina a fonty
\usepackage[czech]{babel}
\usepackage[utf8]{inputenc}
\usepackage[T1]{fontenc}
% dalsi balicky
\usepackage{listings}
\usepackage{graphicx}
\usepackage{enumitem}
\usepackage{indentfirst}
\usepackage{url}
\usepackage[bookmarksopen,colorlinks,plainpages=false,urlcolor=blue,
unicode,linkcolor=black]{hyperref}


\begin{document}

  \begin{titlepage}
    \begin{center}
      \Huge
      \textsc{Fakulta informačních technologií \\
              Vysoké učení technické v~Brně}
      \vspace{100px}
      \begin{figure}[!h]
        \centering
        \includegraphics[height=5cm]{logo}
      \end{figure}
      \\[50mm]
      \LARGE{Dokumentace k projektu předmětu IFJ \,--\, XMPP/Jabber klient}
      \vfill
    \end{center}
    \Large{Roman Blanco (xblanc01) \hfill 29.11.2014}

  \end{titlepage}

  \pagestyle{plain}
  \thispagestyle{empty}
  \pagenumbering{gobble}
  \tableofcontents
  \newpage

  \pagestyle{plain}
  \pagenumbering{arabic}
  \setcounter{page}{3}
  
  \setlength{\parindent}{1cm}
  \setlength{\parskip}{0.5cm plus4mm minus3mm}

  \section{Úvod} \label{uvod}

    Cílem zadaného projektu bylo navrhnout a vytvořit klienta, umožňujícího
    komunikaci prostřednictvím XMPP protokolu. Klient by měl mít schopnosti 
    přijímat a odesílat zprávy, přidávat nové a zobrazovat stávající kontakty
    v seznamu kontaktů, registrovat nového uživatele v síti a také pracovat v
    interaktivním režimu.

  \section{XMPP/Jabber protokol} \label{xmpp}

    Při použití XMPP protokolu je komunikace zapouzdřována do XML zpráv,
    podrobně
    popsaných v několika RFC, které jsou dále zpracovávány serverem či
    klientem. Pro možnost komunikace je potřeba, kromě klienta schopného
    zpracovat zprávy protokolu, mít zaregistrovaný účet na serveru, který se
    nazývá Jabber ID (často používaná zkratka je JID) a má tvar jako běžné
    emailové adresy. Standartně servery XMPP komunikují na portu 5222.

  \section{Popis implementace} \label{implementace}

    Pro implementaci klienta byl použit skriptovací jazyk Python pro jeho
    jednoduchost a dostupnost knihoven potřebných pro vypracování zadaného
    projektu.

    \subsection{Zpracování parametrů} \label{parametry}

      Počáteční fází implementace bylo zpracování parametrů příkazového řádku.
      K tomuto účelu byla použita standardní knihovna jazyka
      Python \,--\, \texttt{argparse}. Díky použití knihovny stačí nastavit
      ve skriptu názvy přepínačů a typ ukládané hodnoty. Knihovna také
      zajišťuje částečné ověření parametrů. Veškeré zpracování parametrů
      příkazového řádku probíhá ve třidě \texttt{Params}.

    \subsection{Připojení k serveru} \label{pripojeni}

      Jednou z nejpodstatnějších částí projektu je komunikace se serverem, bez
      níž by nebylo možné v implementaci pokračovat. Komunikace klienta se 
      serverem je vyřešena ve třídě \texttt{Connection} a byla pro ni 
      použita knihovna \texttt{socket}.
      Pomocí metod knihovny se klient připojí na zadaný server, odesílá zprávy
      od klienta a přijímá zprávy přicházející od serveru.

    \subsection{Zpracování XML zpráv} \label{zpracovani-xml}

      Pro umožnění komunikace mezi klientem a serverem bylo potřeba zajistit
      zpracování přijímaných zpráv. Komunikace XMPP protokolu je
      implementována jako stream. Inicializace XMPP streamu začíná uvozující
      značkou \texttt{<stream:stream>}, a ukončující značka
      \texttt{</stream:stream>} je zaslána až při zakončení komunikace s
      XMPP serverem
      (viz \href{http://xmpp.org/rfcs/rfc3920.html#streams}{RFC 3920}).
      Použitím běžného parsování by došlo k ukončení chybou ve chvíli, kdy by
      parser nenalezl ukončující značku. Z toho důvodu bylo pro zpracování XML
      zpráv potřeba použít iterativní parsování. Parsování a vytváření XML
      zpráv je implementováno ve třídě \texttt{Communication}.

    \subsection{Komunikace s XMPP serverem}

      Vytváření a zpracování obsahu pro komunikaci s XMPP serverem se odehrává
      ve třídě \texttt{Xmpp}.
      První metoda této třídy, která se spustí při komunikaci
      s XMPP serverem je \texttt{init()}. Slouží k inicializaci XML streamu
      a načtení možností pro přihlášení se k účtu. Pro přihlášení slouží
      metoda \texttt{login()}. Ta prostřednictvím jednoho z přihlašovacích
      mechanizmů ověří přihlašovací údaje uživatele,
      a buďto uživatele přihlásí, nebo aplikaci ukončí.
      Dále jsou implementovány metody pro načtení a vypsaní seznamu kontaktů
      (\texttt{loadContacts()}, \texttt{showContacts()}), přidání uživatele do
      seznamu kontaktů (\texttt{addContact()}), registraci nového uživatele
      (\texttt{register()}), přijetí chat zprávy od serveru
      (\texttt{recvMsg()}) a pro odhlášení a ukončení streamu
      (\texttt{logout()}, \texttt{(close())})

    \subsection{Běh programu a interaktivní mód}

      Volání jednotlivých metod pro komunikaci s XMPP probíhá ve třídě
      \texttt{Application} v pořadí, jež je dáno zadáním, a lze je také
      kombinovat pomocí parametrů příkazového řádku. Pokud je program spuštěn
      v interaktivním módu, jsou tyto metody volány podle příkazu uživatele.
      V interaktivním módu má uživatel k dispozici příkazy:

      \begin{description}
        \item[login] přihlášení, pokud byl interaktivní mód spuštěn bez
                     poskytnutí přihlašovacích údajů
        \item[status] vypsaní zda je uživatel přihlášený nebo ne
        \item[whoami] zobrazení JID uživatele
        \item[names] zobrazení seznamu kontaktů uživatele
        \item[send] odeslání chat zprávy jinému uživateli
        \item[exit] ukončení interaktivního módu
        \item[help] vypsaní nápovědy
      \end{description}

  \section{Zajímavější pasáže implementace}

    Zajímavá část implementace je u interaktivního režimu. Jelikož zde mohou
    zprávy přicházet kdykoliv, je potřeba zajistit neustálé načítání ze
    serveru. To je řešeno pomocí vytvoření dalšího vlákna, které má za úkol
    pouze načítat příchozí zprávy.

    Protože však je zároveň potřeba mít
    možnost načíst zpětnou vazbu od serveru po odeslání požadavku, je nutno
    zajistit, aby tuto zpětnou vazbu nepřevzala právě metoda běžící ve
    druhém vlákně. Řešení je implementováno tak, že se odpovědi dočasně
    ukládají do seznamu spolu s transakčním identifikátorem který je u zpětné
    vazby serveru stejný, jako u odeslaného požadavku. Podle identifikátoru se
    poté vrátí odpověď tomu vláknu, kterému patří a odstraní se ze seznamu.

  \section{Použité knihovny jazyka Python}

    Při implementaci projektu byly použity tyto standardní knihovny jazyka
    Python:

    \begin{description}
      \item[argparse] načtení parametrů příkazové řádky
      \item[sys] kvůli zápisu na \texttt{STDERR} při chybě
      \item[logging] k logování chyb a příchozích a odchozích zpráv 
      \item[re] ke kontrole zda zadané JID odpovídají správnému formátu
      \item[socket] k zajištění spojení se serverem
      \item[errno] pro odchycení \texttt{EINTR} \,--\, Interrupted system call
      \item[xml.etree.ElementTree] pro parsování a vytváření XML zpráv
      \item[base64] pro zakódování přihlašovacích údajů
      \item[time] pro metodu \texttt{sleep()} k uspaní programu na zadaný čas
      \item[threading] pro použití vláken
    \end{description}

  \section{Návod k použití}
    
    Pro spuštění je potřeba mít nainstalovaný interpret jazyka Python ve verzi
    3.4. Práva na spouštění obstará Makefile přiložený ke skriptu.

    \noindent
    Ukázka spuštění a přihlášení prostředníctvím interaktivního režimu:
    \begin{lstlisting}
      ./xmppclient -i
      >> help 
      login    [server (port) username password]
      status   []
      whoami   []
      names    []
      send     [receiver message]
      exit     []
      help     []
      >> login netfox.fit.vutbr.cz user password
      >> whoami
      user@netfox.fit.vutbr.cz
      >> status
      Logged in
      >> exit
    \end{lstlisting}

    \noindent
    Ukázka spuštění a zobrazení kontaktů přihlášeného uživatele:
    \begin{lstlisting}
      ./xmppclient -s netfox.fit.vutbr.cz -l user:password -c
      - None <test@netfox.fit.vutbr.cz>
      - New Buddy <anyone@netfox.fit.vutbr.cz>
      - None <zkouska@netfox.fit.vutbr.cz>
    \end{lstlisting}

  \section{Závěr}

  Vzhledem k tomu, že jazyk Python je multiplatformní a k implementaci nebyly
  použity knihovny třetích stran, měl by být klient spustitelný na jakékoliv
  platformě s touto verzí interpretu.

  Výsledný program lze použít k jednoduché komunikaci, nicméně díky nízké
  uživatelské přístupnosti by bylo vhodnější řešení využít program
  ke komunikaci mezi několika programy.

  Díky rozdělení kódu do logických struktur by mělo být možné kód jednoduše
  editovat a přizpůsobit požadovaným účelům.

  \section{Zdroje}

    \begin{enumerate}[label={[\arabic*]}]
      \item RFC 3920 \href{https://tools.ietf.org/html/rfc3920}
                          {tools.ietf.org/html/rfc3920}
      \item RFC 3921 \href{https://tools.ietf.org/html/rfc3921}
                          {tools.ietf.org/html/rfc3921}
      \item RFC 6120 \href{https://tools.ietf.org/html/rfc6120}
                          {tools.ietf.org/html/rfc6120}
      \item RFC 6121 \href{https://tools.ietf.org/html/rfc6121}
                          {tools.ietf.org/html/rfc6121}
      \item Dokumentace jazyka Python 3.4 \href{https://docs.python.org/3.4/}
                                               {docs.python.org/3.4}

    \end{enumerate}

\end{document}
