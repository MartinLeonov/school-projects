  \documentclass[11pt,a4paper]{article}
% rozmery stranky
\usepackage[left=1.5cm,text={18cm, 25cm},top=2.5cm]{geometry}
% cestina a fonty
\usepackage[czech]{babel}
\usepackage[utf8]{inputenc}
\usepackage[T1]{fontenc}
% dalsi balicky
\usepackage{graphicx}
\usepackage{enumitem}
\usepackage{indentfirst}
\usepackage{url}
\usepackage[bookmarksopen,colorlinks,plainpages=false,urlcolor=blue,
            unicode,linkcolor=black]{hyperref}

\begin{document}

  \begin{titlepage}
    \begin{center}
      \Huge
      \textsc{Fakulta informačních technologií\\ Vysoké učení technické v~Brně}
      \vspace{100px}
      \begin{figure}[!h]
        \centering
        \includegraphics[height=5cm]{logo.jpg}
      \end{figure}
      \\[50mm]
      \LARGE{Dokumentace k~projektu předmětu ISJ \,--\, Skriptovací jazyky
             \newline
             Stahování dat z~Twitteru a diskusního fóra}
      \vfill
    \end{center}
    \Large{Roman Blanco (xblanc01) \hfill 2.4.2015}
  \end{titlepage}

  \tableofcontents
  \newpage

  \section{Úvod}

    Dokumentace pojednává o~skriptu napsaném v~rámci předmětu Skriptovací
    jazyky.

    Zadání bylo rozděleno na dvě části. Jednou z~částí bylo napsat skript,
    který stáhne všechny dosavadní příspěvky diskuzního fóra a uloží je lokálně
    do souboru společně s~relevantními metainformacemi. Skript by také měl být
    schopen rozpoznat nové příspěvky a zahájit jejich stahování.
    Druhá část zadání spočívala v~úkolu napsat skript, jež bude sloužit ke
    sledování účtu na sociální síti Twitter. Příspěvky mají být dle zadání
    lokálně ukládány a mají být nalezeny případné odkazy na webové stránky a
    stažen jejich obsah.

    Pro jednoduchost a fakt, že v~obou částech řešení jsou využity totožné
    metody (práce s~JSON soubory, zpracování parametrů, řešení chybových
    stavů), jsou obě části zadání implementovány v~rámci jednoho souboru,
    s~tím, že kontext pro práci s~fórem a se~sítí Twitter se přepíná pomocí
    parametru, jak je popsáno níže, nebo také v~nápovědě skriptu, kterou lze
    vyvolat spuštěním s~parametrem \texttt{-h} nebo \texttt{--help}.

  \section{Twitter}

    Rozhraní skriptu pro práci se sociální sítí Twitter je specifikováno pomocí
    parametru \texttt{twitter} uvedeným za přepínačem \texttt{-i}.
    Akce, která má být vykonána je specifikována parametrem \texttt{-a}.

    \subsection{Načtení nových dat} \label{twitter-nacteni}

      Je-li skript spuštěn s~parametrem \texttt{load} za přepínačem
      \texttt{-a}, dojde ke stažení tzv. \textit{tweetů} vytvořených sledovaným
      účtem.

      Úvodním krokem je připojení se k~API sítě Twitter a načtení zpráv
      uživatele. Pro umožnění připojení k~API a úspěšné autentifikaci aplikace
      je použita knihovna třetí strany
      \href{https://github.com/oauth-xx/oauth-ruby}{oauth}, jejiž použití
      pro autentifikaci je popsáno na webových stránkách Twitteru určených pro
      vývojáře.
      Tuto knihovnu je možné nainstalovat jako Ruby Gem příkazem
      \texttt{gem install oauth}

      Po zaslání požadavku na
      \href{https://dev.twitter.com/rest/reference/get/statuses/user_timeline}
      {Twitter API} je skriptu poslána zpět odpověď ve formátu JSON, kterou
      následně skript parsuje a vyfiltruje z~ní pouze požadované údaje
      (identifikátor tweetu, datum a čas vytvoření, text, URL adresy obsažené
      v~textu). Tyto načtené zprávy jsou následně porovnány s~případně již
      staženými, a jsou-li nalezeny novější, jsou přidány do souboru.

    \subsection{Zobrazení obsahu lokálně uložených dat}

      Při zadání parametru \texttt{show} za přepínačem \texttt{-a} skript
      zobrazí soubor s~načtenými tweety. Zobrazeny jsou již výše zmíněné údaje:

      \begin{itemize}
        \item identifikátor tweetu
        \item datum a čas vytvoření
        \item text tweetu
        \item URL adresy obsažené v~textu
      \end{itemize}

      O~výpis souboru se stará externí knihovna
      \href{https://github.com/michaeldv/awesome_print}{awesome\_print}, což je
      přehlednější modifikace standardní knihovny PrettyPrint. Knihovnu
      lze nainstalovat jako Gem příkazem \texttt{gem install awesome\_print}.

    \subsection{Stažení URL adres}

      \subsubsection{Bez poskytnuti identifikátoru}

        Při spuštění skriptu s~parametrem \texttt{download}, přičemž není dále
        specifikovaný identifikátor tweetu, se nejprve vykonají tytéž úkony,
        jako při spuštění s~parametrem \texttt{load} (viz kapitola
        \ref{twitter-nacteni}) a poté dojde ke stažení obsahu všech nalezených
        URL odkazů v~nově načtených tweetech.

      \subsubsection{S~identifikátorem}

        Pokud je s~parametrem \texttt{download} zadaný i indentifikátor pomocí
        přepínače \texttt{-s}, pokusí se skript vyhledat zprávu s~tímto
        identifikátorem mezi tweety v~lokálně uloženém souboru. Pokud případně
        nalezený tweet obsahuje alespoň jednu URL adresu, skript stáhne její
        obsah.

      Stažené odkazy mají vždy jako součást prefixu identifikační číslo tweetu,
      z~něhož byly načteny, díky čemuž je možné lokálně uložit obsah z~více
      zdrojů při jediném načtení timeline uživatele.

    \subsection{Specifikovaný účet}

      Skript je rozšířen o~možnost specifikování jiného, než defaultně
      zvoleného účtu, který je zapsán přímo ve zdrojovém kódu.
      Pokud je pomocí přepínač \texttt{-c} zadaný twitter účet, skript bude
      pracovat právě se specifikovaném účtem. Při načtení tweetu do lokalního
      souboru je součástí prefixu souboru take název sledovaného účtu, čímž je
      vyřešen problem s~přepsaním načtených dat v~lokálním souboru vyřešen.

  \section{Fórum}

    Stejně tak, jako je rozhraní pro práci se sítí Twitter specifikováno
    parametrem \texttt{twitter}, je rozhraní pro práci s~diskuzním fórem
    specifikováno parametrem \texttt{forum} za přepínačem \texttt{-i}.

    Nutným požadavkem při práci s~fórem je zadání URL adresy požadovaného
    vlákna pomocí přepínače \texttt{-s} a operace, která se má provést
    přepínačem \texttt{-a}. Z~adresy je pak regulárním výrazem vyfiltrováno
    identifikační číslo vlákna, které slouží k~odlišení lokálně uloženého
    souboru, pro případ, že by uživatel měl zájem sledovat skriptem více vláken
    současně.

    \subsection{Načtení dat}

      Vybrané fórum \href{http://talk.manageiq.org}{talk.manageiq.org} je
      provozováno aplikací \href{http://www.discourse.org}{Discourse},
      která nabízí v~tuto chvíli ještě nepříliš propracované API,
      avšak dostačující ke získání potřebných dat z~fóra.
      Ukázka práce s~API byla demonstrována
      v~\href{https://meta.discourse.org/t/discourse-api-documentation/22706}
      {příspěvku} na diskuzním fóru aplikace.

      Při spuštění skriptu s~parametrem \texttt{load} dojde k~načtení
      dat z~API zadané URL adresy vlákna, z~něhož jsou následně vyfiltrovaná
      požadovaná data příspěvku v~daném vláknu:

      \begin{itemize}
        \item počet příspěvků ve vláknu
        \item identifikátor příspěvku
        \item autor příspěvku
        \item text příspěvku
      \end{itemize}

      Podle načteného počtu příspěvku ve vláknu se pak určí, zda byly od
      posledního načtení vlákna do lokálního souboru přidány nějaké nové
      příspěvky a zapíší se do již vytvořeného lokálního souboru, případně
      bude vytvořen nový soubor.

    \subsection{Zobrazení dat}

      Funkce parametru \texttt{show} je totožná jako při práci se sociální
      sítí Twitter. Lokálně uložená data jsou vypsány pomocí knihovny
      \href{https://github.com/michaeldv/awesome_print}{awesome\_print}.
      Pokud lokální soubor pro dané vlákno ještě neexistuje, je na to uživatel
      upozorněn chybovým hlášením.

\end{document}
