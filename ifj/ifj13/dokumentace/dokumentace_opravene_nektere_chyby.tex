\documentclass[12pt,a4paper,titlepage,final]{article}

% cestina a fonty
\usepackage[czech]{babel}
\usepackage[utf8]{inputenc}
\usepackage[T1]{fontenc}
\usepackage{lmodern}
\usepackage{textcomp}

% odsazeni prvniho radku
\usepackage{indentfirst}

% balicky pro odkazy
\usepackage[bookmarksopen,colorlinks,plainpages=false,urlcolor=blue,
unicode,linkcolor=black]{hyperref}
\usepackage{url}

% obrazky
\usepackage[dvipdf]{graphicx}

% velikost stranky
\usepackage[top=3.5cm, left=2.5cm, text={17cm, 24cm}, ignorefoot]{geometry}

% cislovani (pouzito kvuli cislovani u pouzite literatury a zdroju)
\usepackage{enumitem}

\begin{document}

%%%%%%%%%%%%%%%%%%%%%%%%%%%%%%%%%%%%%%%%%%%%%%%%%%%%%%%%%%%%%%%%%%%%%%%%%%%%%%

  % uvodni strana
  \begin{titlepage}

    \centering
    \includegraphics[height=5cm]{fitlogo.eps}

    \vfill

    \begin{center}
      \begin{Large}
        Dokumentace k~projektu pro předměty IFJ a IAL \\
      \end{Large}
      \bigskip
      \begin{LARGE}
        Implementace interpretu imperativního jazyka IFJ13 \\[1cm]
      \end{LARGE}
      \begin{large}
        Tým 099, varianta a/2/I
      \end{large}
    \end{center}

    \vfill

    \begin{center}
      \begin{Large}
        \today
      \end{Large}
    \end{center}

    \vfill

    \begin{flushleft}
      \subsubsection*{Autoři:}
        \begin{tabular}{llr}
          vedoucí: Michael Vlček & \texttt{xvlcek21} & 20\% \\
          Radim Sváček           & \texttt{xsvace02} & 20\% \\
          Roman Blanco           & \texttt{xblanc01} & 20\% \\
          Vojtěch Mička          & \texttt{xmicka06} & 20\% \\
          Richard Wolfert        & \texttt{xwolfe00} & 20\% \\
        \end{tabular}
    \end{flushleft}

  \end{titlepage}\setlength{\parindent}{1cm}

%%%%%%%%%%%%%%%%%%%%%%%%%%%%%%%%%%%%%%%%%%%%%%%%%%%%%%%%%%%%%%%%%%%%%%%%%%%%%%

  % obsah
  \pagestyle{plain}
  \thispagestyle{empty}
  \pagenumbering{gobble}
  \tableofcontents
  \newpage

%%%%%%%%%%%%%%%%%%%%%%%%%%%%%%%%%%%%%%%%%%%%%%%%%%%%%%%%%%%%%%%%%%%%%%%%%%%%%%

  % nastaveni cislovani
  \pagestyle{plain}
  \pagenumbering{arabic}
  \setcounter{page}{3}
  
  % nastaveni mezery mezi odstavci a odsazeni prvniho radku
  \setlength{\parindent}{1cm}
  \setlength{\parskip}{0.5cm plus4mm minus3mm}

%%%%%%%%%%%%%%%%%%%%%%%%%%%%%%%%%%%%%%%%%%%%%%%%%%%%%%%%%%%%%%%%%%%%%%%%%%%%%%
  \section{Úvod} \label{uvod}
%%%%%%%%%%%%%%%%%%%%%%%%%%%%%%%%%%%%%%%%%%%%%%%%%%%%%%%%%%%%%%%%%%%%%%%%%%%%%%

    Tato dokumentace pojednává o~interpretu imperativního jazyka IFJ13,
    zadaného jako skupi\-no\-vý projekt pro předmety IFJ a IAL. Níže najdete
    popis implementace některých základních částí tohoto projektu, jako
    náčrt konečného automatu pro lexikální analýzu, LL gramatiku syntaktického
    analyzátoru, postupy některých předepsaných metod a algoritmů pro
    využívané prvky či samotného interpretu. Dále pak rozdělení a koordinace
    práce v~týmu a seznam využitých zdrojů informací.
    \newpage

%%%%%%%%%%%%%%%%%%%%%%%%%%%%%%%%%%%%%%%%%%%%%%%%%%%%%%%%%%%%%%%%%%%%%%%%%%%%%%
  \section{Lexikální analýza} \label{lexikalni-analyza}
%%%%%%%%%%%%%%%%%%%%%%%%%%%%%%%%%%%%%%%%%%%%%%%%%%%%%%%%%%%%%%%%%%%%%%%%%%%%%%

    Implementace lexikálního analyzátoru je řešena jako konečný automat, jež
    načítá znaky ze zdrojového souboru. Na základě těchto znaků vytváří
    lexikální analyzátor tokeny, navíc souběžně vrací číselnou hodnotu
    popisující daný token. Tokeny jsou reprezentovány abstraktním datovým
    typem, který krom samotného tokenu, jež je uložen jako řetězec, uchovává
    navíc i ukazatel na poslední znak a informaci o~celkovém počtu alokované
    paměti. Pro omezení počtu alokací je paměť alokována po 8 bytech. 
    V~případě, že neexistuje z~daného stavu konečného automatu přechod
    do~libovolného dalšího stavu, vrací lexikální analyzátor informaci 
    o~lexikální chybě ve zdrojovém programu. 

%%%%%%%%%%%%%%%%%%%%%%%%%%%%%%%%%%%%%%%%%%%%%%%%%%%%%%%%%%%%%%%%%%%%%%%%%%%%%%
  \section{Syntaktická analýza} \label{syntakticka-analyza}
%%%%%%%%%%%%%%%%%%%%%%%%%%%%%%%%%%%%%%%%%%%%%%%%%%%%%%%%%%%%%%%%%%%%%%%%%%%%%%

    \subsection{Parser}

    Syntaktická analýza tvoří základ každého překladače. Hlavní úlohou
    syntaktické analýzy je zjistit, zda je zdrojový kód zapsán syntakticky
    správně. V~tomto projektu jsme na analýzu základních programovacích
    konstrukcí použili rekurzivní syntaktickou analýzu shora dolů. Tuto
    metodu jsme implementovali s~použitím níže popsané LL gramatiky.

    \subsection{Precedenční syntaktická analýza}

    V~této části syntaktického analyzátoru je nejprve předáno řízení
    vyhodnocování výrazu precedenční syntaktické analýze (\textit{PSA}), načež
    je vyhodnocen samotný výraz. Pro kontrolu správnosti výrazu z~hlediska
    syntaxe je oproti klasickému řešení pomocí zásobníku a derivačního stromu
    potřeba využit dvousměrně vázaný seznam, který umožňuje skloubit zásobník
    a derivační strom (potažmo abstraktní derivační strom) do jedné struktury.
    Součástí syntaktické analýzy z~hlediska struktur je i tabulka symbolů a
    statická tabulka přechodů pro určení priority a asociativity operátorů
    (případně zde můžeme zahrnout i výstupní seznam instrukcí).

    \subsection{Dvousměrně vázaný seznam}

    Seznam funguje podobně jako zásobník (na začátku seznamu po inicializaci
    ukončovací znak \texttt{\$}, vkládání prioritně/asociativních znaků
    \texttt{<}, ...). Veškeré prvky dvousměrně vázaného seznamu (včetně
    terminálů) jsou uloženy ve formě instrukcí, stejným způsobem jako
    instrukce ve výstupním seznamu. V~případě načtení terminálu je do seznamu
    přidána nová \clqq instrukce \crqq, jejíž oba ukazatele na operandy jsou
    hodnoty \texttt{NULL} a ostatní atributy (typ instrukce, ukazatel na
    výsledek) jsou přiřazeny následovně podle typu terminálu:

    \begin{enumerate}
      \item \label{i} Pokud se jedná o~proměnnou, typ instrukce
      \texttt{INST\_NOP} (prázdná instrukce, použita pouze pro reprezentaci
      v~seznamu, nebude přidána do výstupního seznamu instrukcí) a ukazatel na
      výsledek ukazuje do tabulky symbolů proměnných na místo, kde je uložena
      příslušná proměnná. Pokud neexistuje, je vrácena chyba.

      \item Pokud jde o~konstantu, vytvoří se nová položka v~tabulce symbolů
      proměnných, jejíž klíč je řetězcová reprezentace této konstanty. Dále se
      postupuje analogický jako u~bodu \ref{i}.

      \item Pokud jde o~znaménko nebo levou závorku, typ instrukce je hodnota
      tohoto znaménka v~naší enumeraci, ukazatel na výsledek je \texttt{NULL}.
    \end{enumerate}

    Terminální symboly gramatiky jsou postupně nasouvány do seznamu podobně
    jako u~zásobníku. Aplikace redukce podle pravidel funguje také podobně,
    ale při aplikaci pravidla je po kontrole jeho syntaxe rovnou možné přidat
    instrukci na konec výstupního instrukčního seznamu. Pro ilustraci našeho
    řešení uvádím příklad provádění redukce sčítání těsně před redukcí sčítání
    a po ní:

    \noindent
    \textit{Stav obou seznamů před načtením posledního znaku '\texttt{;}',
    který značí návrat řízení analýze rekurzivního sestupu:}

    \begin{small}
      \noindent
      \begin{tabular}{|c|l|c|c|c||l|l|c|c|c|}

        \multicolumn{5}{c}{Dvousměrně vázaný seznam} &
        \multicolumn{5}{c}{Seznam instrukcí} \\

        \hline

        abstr. & instrukce & adr1 & adr2 & result &
        abstr. & instrukce & adr1 & adr2 & result \\

        \hline

        \texttt{'\$'} & \texttt{semicolon} & \texttt{NULL} &
        \texttt{NULL} & \texttt{NULL} & 
        C $\rightarrow$ i & \texttt{INST\_ASSIGN} & \texttt{GEN0} &
        \texttt{NULL} & \texttt{GEN0} \\

        \texttt{'<'} & \texttt{LEFT\_ARROW} & \texttt{NULL} &
        \texttt{NULL} & \texttt{NULL} &
        C $\rightarrow$ i & \texttt{INST\_ASSIGN} & \texttt{GEN1} &
        \texttt{NULL} & \texttt{GEN1} \\

        C $\rightarrow$ i & \texttt{INST\_ASSIGN} & \texttt{GEN0} &
        \texttt{NULL} & \texttt{GEN0} & & & & & \\

        \texttt{'+'} & \texttt{plus} & \texttt{NULL} &
        \texttt{NULL} & \texttt{NULL} & & & & & \\

        C $\rightarrow$ i & \texttt{INST\_ASSIGN} & \texttt{GEN1} &
        \texttt{NULL} & \texttt{GEN1} & & & & & \\

        \hline

      \end{tabular}
    \end{small}

    \noindent
    \textit{Po načtení znaku se přidá podle statické tabulky přechodů znak 
    '\texttt{>}'}:

    \begin{small}
      \noindent
      \begin{tabular}{|c|l|c|c|c||l|l|c|c|c|}

        \multicolumn{5}{c}{Dvousměrně vázaný seznam} &
        \multicolumn{5}{c}{Seznam instrukcí} \\

        \hline
        abstr. & instrukce & adr1 & adr2 & result &
        abstr. & instrukce & adr1 & adr2 & result \\

        \hline

        \texttt{'\$'} & \texttt{semicolon} & \texttt{NULL} &
        \texttt{NULL} & \texttt{NULL} &C $\rightarrow$ i &
        \texttt{INST\_ASSIGN} & \texttt{GEN0} & \texttt{NULL} &
        \texttt{GEN0} \\

        \texttt{'<'} & \texttt{LEFT\_ARROW} & \texttt{NULL} &
        \texttt{NULL} & \texttt{NULL} & C $\rightarrow$ i &
        \texttt{INST\_ASSIGN} & \texttt{GEN1} & \texttt{NULL} &
        \texttt{GEN1} \\

        C $\rightarrow$ i & \texttt{INST\_ASSIGN} & \texttt{GEN0} &
        \texttt{NULL} & \texttt{GEN0} & & & & & \\

        \texttt{'+'} & \texttt{plus} & \texttt{NULL} & \texttt{NULL} &
        \texttt{NULL} & & & & & \\

        C $\rightarrow$ i & \texttt{INST\_ASSIGN} & \texttt{GEN1} &
        \texttt{NULL} & \texttt{GEN1} & & & & & \\

        \texttt{'>'} & \texttt{RIGHT\_ARROW} & \texttt{NULL} & \texttt{NULL} &
        \texttt{NULL} & & & & &  \\

        \hline

      \end{tabular}
    \end{small}

    \noindent
    \textit{Provede se redukce C → C + C:}

    \begin{small}
      \noindent
      \begin{tabular}{|l|l|c|c|c||l|l|c|c|c|}

        \multicolumn{5}{c}{Dvousměrně vázaný seznam} &
        \multicolumn{5}{c}{Seznam instrukcí} \\

        \hline

        abstr. & instrukce & adr1 & adr2 & result &
        abstr. & instrukce & adr1 & adr2 & result \\

        \hline

        \texttt{'\$'} & \texttt{semicolon} & \texttt{NULL} & \texttt{NULL} &
        \texttt{NULL} & C $\rightarrow$ i & \texttt{INST\_ASSIGN} &
        \texttt{GEN0} & \texttt{NULL} & \texttt{GEN0} \\

        C$\rightarrow$C+C & \texttt{INST\_PLUS} & \texttt{GEN0} &
        \texttt{GEN1} & \texttt{GEN2} & C $\rightarrow$ i &
        \texttt{INST\_ASSIGN} & \texttt{GEN1} & \texttt{NULL} &
        \texttt{GEN1} \\

        & & & & & C$\rightarrow$C+C & \texttt{INST\_PLUS} & \texttt{GEN0} &
        \texttt{GEN1} & \texttt{GEN2} \\

        \hline

      \end{tabular}
    \end{small}

%%%%%%%%%%%%%%%%%%%%%%%%%%%%%%%%%%%%%%%%%%%%%%%%%%%%%%%%%%%%%%%%%%%%%%%%%%%%%%
  \section{Sémantická analýza} \label{semanticka-analyza}
%%%%%%%%%%%%%%%%%%%%%%%%%%%%%%%%%%%%%%%%%%%%%%%%%%%%%%%%%%%%%%%%%%%%%%%%%%%%%%

    Sémantická analýza je v~našem projektu řešena částečně v~syntaktickém
    analyzátoru a částečně v~interpretu. Při syntaktické analýze jsou
    detekovány statické sémantické chyby a interpret má na starost detekci
    běhových či sémantických chyb až při samotné interpretaci programu.

%%%%%%%%%%%%%%%%%%%%%%%%%%%%%%%%%%%%%%%%%%%%%%%%%%%%%%%%%%%%%%%%%%%%%%%%%%%%%%
  \section{Algoritmy pro práci s~řetězci} \label{algoritmy}
%%%%%%%%%%%%%%%%%%%%%%%%%%%%%%%%%%%%%%%%%%%%%%%%%%%%%%%%%%%%%%%%%%%%%%%%%%%%%%

    \subsection{Implementace řadícího algoritmu}

    Ze~zadání vyplývala povinnost implementovat vestavěnou funkci pro řazení
    řetězce pomocí algoritmu Heapsort. Jedná se o~řadící algoritmus
    s~linearitmetickou složitostí. Základem tohoto algoritmu je halda, která
    je reprezentována samotným řetězcem, který je vstupním parametrem funkce.
    Díky tomu nemá algoritmus žádné výraznější nároky na paměť.

    Funkce nejprve ustaví haldu, jež je přímo ve~vstupním řetězci. Následně se
    provádí odebírání vrcholu haldy a jeho záměna s~posledním prvkem. Poté
    následuje rekonsturkce již zkrácené haldy, která opět ustaví haldu do
    korektní podoby. Tento cyklus se provádí, dokud jsou prvky k~řazení.

    Rekontrukce haldy probíhá v~cyklu, kdy je třeba ověřit, zda je otcovský
    uzel větší než oba synovské uzly. V~opačném případě dojde k~jejich záměně,
    přičemž se musí pokračovat i s~haldou, jejíž otcovský uzel je nyní
    aktuálně vyměněný prvek. Tento jev se opakuje, dokud není dosaženo
    pravidla haldy.

    \subsection{Implementace vyhledávacího algoritmu}

    Stejně jako v~případě řadícího algoritmu, byl i algoritmus pro vyhledávání
    v~poli určen zadáním. Pro implementaci bylo tedy nutné využít
    Knuth–Morris–Prattův algoritmus řazení pole. Ten je založený na myšlence,
    kdy je zbytečné porovnávat znaky, jež již porovnány byly. K~tomu, aby bylo
    možno omezit opětovné porovnávání znaků je nutné implementovat pole hodnot
    pro každou pozici řetězce, které označují index řetězce, na který se
    funkce při neúspěšném porovnání vrátí a bude porovnávat další znaky
    s~dalšími znaky zdrojového řetězce. Vstupem funkce je kromě hledaného
    řetěze i zdrojový řetězec. Funkce vrací index zdrojového řetězce, na
    kterém byla nalezena shoda.

    Nejprve je nutné vytvořit pole hodnot, které je vytvořeno na základě
    hledaného řetězce. V~něm se můžou vyskytovat opakující se sekvence znaků,
    které je možno přeskočit. Například při řetězci \textsl{ABABC} není nutné
    opakovaně porovnávaly znaky \textsl{AB}, ale při neshodě na pátém znaku je
    možné se vrátit na třetí znak a pokračovat. Výsledné pole tak vypadá
    následovně -- [-1,0,0,1,2].

    Poté již přichází na řadu samotné vyhledávání. V ideálním případě se při
    shodě s~prvním znakem hledaného řetězce začnou porovnávat další znaky až
    do dosažení konce hledaného řetězce, načež se vrátí index shody s~prvním
    znakem. Pokud ovšem kdykoliv během porovnávání narazí na neshodu znaků,
    obnoví se index v~hledaném řetězci tak, že se k~hodnotě počátku shody
    přičte hodnota z~pomocného pole na indexu neshody. Díky tomu se omezí
    počet porovnávání již porovnaných znaků a výsledný proces je rychlejší.

%%%%%%%%%%%%%%%%%%%%%%%%%%%%%%%%%%%%%%%%%%%%%%%%%%%%%%%%%%%%%%%%%%%%%%%%%%%%%%
  \section{Způsob a řešení realizace tabulky symbolů} \label{tabulka-symbolu}
%%%%%%%%%%%%%%%%%%%%%%%%%%%%%%%%%%%%%%%%%%%%%%%%%%%%%%%%%%%%%%%%%%%%%%%%%%%%%%

    \subsection{Způsob realizace tabulky symbolů}

    Ze zadání byl určen způsob realizace tabulky symbolů, v~našem případě
    binárním stromem. Nejprve bylo potřeba z~opory a přednášek k~předmětu IAL
    zjistit informace o~binárním stromu, tedy že se jedná o~strom prvků, kde
    každý prvek, krom posledních, má dva ukazatele na levý a pravý podstrom.
    V~levém podstromu se pak uchovávají hodnoty nižší než jejich kořenový
    prvek a v~pravém podstromu hodnoty vyšší.

    \subsection{Binární strom}

    Prvním důležitým bodem po získání informací o~binárním stromu bylo
    ujasnění si, k~čemu tabulka symbolů slouží a jaké operace nad ní budou
    požadovány. Jelikož v~každém prvku bude potřeba uchovávat více informací,
    zvolili jsme jako každou položku stromu strukturu. Dále jsme se rozhodli
    realizovat dvě odlišené tabulku symbolů. Jednu pro uchovávání proměnných
    a druhou pro funkce. To zejména kvůli přehlednosti při jejich používání.

    \subsection{Tabulka symbolů proměnných}

    Do tabulky symbolů proměnných je jako unikátní vyhledávací klíč uvedeno
    jméno proměnné a dva ukazatele na každý podstrom (v~případě
    neexistujícího podstromu nastavené na hodnotu \texttt{NULL}). Dále bylo
    zjevné, že je potřeba uchovat datový typ proměnné a její data. Tyto dva
    prvky jsme se pro přehlednost rozhodli uložit do další struktury a
    uchovávaná data pak realizovat unionem, pro jeho výhodné uložení v~paměti,
    \clqq union zabírá v~paměti jen tolik místa, kolik největší jeho
    položka\crqq. Union obsahuje položky pro uchování datového typu
    \texttt{integer}, \texttt{double}, \texttt{string} a \texttt{boolean}.
    Pro datový typ \texttt{NULL} není třeba mít vyhrazenou položku, neboť
    neuchovává žádná data.

    \subsection{Tabulka symbolů funkcí}

    Tabulka symbolů pro funkce obsahovala nejprve pouze jméno funkce jako
    unikátní vyhledávací klíč, dvojici ukazatelů na podstromy a položku
    uchovávající počet parametrů funkce. S~vývojem parseru však bylo
    zjištěno, že je potřeba přidat ukazatel na seznam instrukcí a také
    ukazatel na tabulku symbolů proměnných. Pro práci s~oběma typy tabulek
    bylo potřeba udělat funkce pro inicializaci, vyhledávání, vkládání a
    zrušení celé tabulky. Funkce pro práci s~tabulkou symbolů pro funkce a
    proměnné jsou v~principu stejné, ale místy se liší ve funkčnosti. Krom
    rozdílných parametrů (kvůli rozdílným prvkům ve strukturách) například
    funkce pro vyhledávání v~tabulce proměnných při shodě klíčů provede změnu
    zadaných prvků, kdežto vyhledávací funkce pro tabulku symbolů funkcí
    vrátí chybu (jelikož v~jazyce IFJ13 nelze přetěžovat funkce).
    Nejzajímavější z~funkcí jsou funkce pro vkládání a rušení tabulky. Funkce
    pro vkládání nejprve v~cyklu projde podle příslušného vyhledávacího klíče
    celý binární strom, za použití funkce \texttt{strcmp(char *s1, char *s2)}.
    \clqq Ta porovnává řetězce \texttt{s1} a \texttt{s2}. Pokud je \texttt{s1}
    $<$ \texttt{s2} pak vrací hodnotu menší než $0$, pokud jsou si rovny, pak
    vrací $0$, pokud je \texttt{s1} $>$ \texttt{s2} pak vrací hodnotu větší
    než $0$.\crqq Pokud nebyla nalezena žádná odpovídající položka, alokuje
    místo pro novou položku, pro řetězec reprezentující vyhledávací klíč
    a pokud se jedná o~tabulku proměnných také datový typ \texttt{string},
    nealokuje paměť i pro tento řetězec. Následně podle datového typu vloží
    příslušná data z~předaných parametrů a vrátí v~případě úspěchu makro pro
    úspěch, jinak makro pro chybu.

    \subsection{Zásobník ukazatelů na položky binárního stromu}

    Při realizaci funkce pro rušení binárního stromu jsme zjistili potřebu
    zásobníku ukazatelů na položky binárního stromu. Zásobník jsme pojali jako
    jednosměrně vázaný seznam struktur a několik funkcí pro práci s~nimi.
    Jedna položka zásobníku vždy obsahuje užitečná data, tedy ukazatel
    na položku tabulky symbolů a ukazatel na předešlý prvek seznamu
    (zásobníku). Jako funkce pro práci s~ním pak stačilo implementovat
    inicializaci, vkládání (\texttt{Push}) a odstranění (\texttt{Pop})
    z~vrcholu zásobníku. Funkce pro přístup k~vrcholu zásobníku není potřeba,
    kvůli snadnému přístupu k~datům ve struktuře. Funkce pro rušení binárního
    stromu pak za použití zásobníku prochází binární strom a postupně ruší
    jednotlivé prvky voláním funkce \texttt{free()}. Rovněž je třeba uvolnit
    paměť nealokovanou pro vyhledávací klíč a v~případě, že se jedná o~tabulku
    proměnných a uchovávaný datový typ byl \texttt{string}, pak i pro něj
    nealokovanou paměť. Při inicializaci tabulky funkcí se rovnou vkládají
    záznamy o~vestavěných funkcích jazyka IFJ13.

    \subsection{Další funkce}

    Během implementace dalších částí interpretu bylo zjištěno, že je potřeba
    dalších funkcí. Mezi funkce pracující s~tabulkou proměnných přibyla
    funkce, která projde a okopíruje existující tabulku do nové. Pro tuto bylo
    třeba napsat pomocnou funkci, která postupně procházela binární strom až
    k~nejlevějšímu prvku, \texttt{letftMost()} a plnila pomocný zásobník
    ukazateli na každý prvek. Pak se skrze zásobník procházel strom odspodu
    nahoru a kopírovala jednotlivé položky. V~každém prvku se také
    kontrolovala existence jeho ukazatele na prvý podstrom, případně se nad
    ním zavolala pomocná funkce \texttt{letftMost()}. Mezi funkce pracující
    s~tabulkou funkcí pak také přibyla funkce která opět za pomoci funkce
    \texttt{letftMost()} procházela strom a kontrolovala existenci ukazatele
    na seznam instrukcí. Pak už jen několik testovacích funkcí kde stačily
    jednoduché průchody binárním stromem a tisknutí položek.

%%%%%%%%%%%%%%%%%%%%%%%%%%%%%%%%%%%%%%%%%%%%%%%%%%%%%%%%%%%%%%%%%%%%%%%%%%%%%%
  \section{Rozdělení práce v~týmu a komunikace} \label{rozdeleni}
%%%%%%%%%%%%%%%%%%%%%%%%%%%%%%%%%%%%%%%%%%%%%%%%%%%%%%%%%%%%%%%%%%%%%%%%%%%%%%

    \subsection{Schůze a komunikace v týmu}

    Díky blízkosti ubytování členů našeho týmu bylo možné konat poradní
    schůze prakticky kdykoliv bylo potřeba, nebyla tedy potřeba obtěžovat se
    s~organizací a zřizováním komunikačních nástrojů.

    \subsection{Správa verzí zdrojového kódu}

    Pro přehlednost vývoje interpretu jsme použili verzovací system Git
    z~důvodu předchozí zkušenosti s~jeho používáním jednoho člena týmu.
    Zdrojové kódy jsme měli uložené na soukromém repozitáři webové služby
    \href{http://bitbucket.org/}{Bitbucket}.
    \newpage

%%%%%%%%%%%%%%%%%%%%%%%%%%%%%%%%%%%%%%%%%%%%%%%%%%%%%%%%%%%%%%%%%%%%%%%%%%%%%%
  \section{Literatura, reference, zdroje informací} \label{literatura}
%%%%%%%%%%%%%%%%%%%%%%%%%%%%%%%%%%%%%%%%%%%%%%%%%%%%%%%%%%%%%%%%%%%%%%%%%%%%%%

    \begin{enumerate}[label={[\arabic*]}]
      \item Skripta k~předmětu IAL
      \item Skripta k~předmětu IFJ
      \item Popis typu \texttt{union}:
      \href{http://www.sallyx.org/sally/c/c16.php}
        {http://www.sallyx.org/sally/c/c16.php}
      \item Popis \texttt{strcmp}: 
        \href{http://www.sallyx.org/sally/c/c19.php}
          {http://www.sallyx.org/sally/c/c19.php}
    \end{enumerate}

%%%%%%%%%%%%%%%%%%%%%%%%%%%%%%%%%%%%%%%%%%%%%%%%%%%%%%%%%%%%%%%%%%%%%%%%%%%%%%

  % prilohy
  \appendix
  \newpage

%%%%%%%%%%%%%%%%%%%%%%%%%%%%%%%%%%%%%%%%%%%%%%%%%%%%%%%%%%%%%%%%%%%%%%%%%%%%%%
  \section{Přílohy} \label{prilohy}
%%%%%%%%%%%%%%%%%%%%%%%%%%%%%%%%%%%%%%%%%%%%%%%%%%%%%%%%%%%%%%%%%%%%%%%%%%%%%%

    \subsection{Konečný automat}

    \includegraphics[height=14cm]{automat.eps}

    \begin{tabular}{ll|ll|ll}
      f1 & \texttt{identificator} & f16 & \texttt{left\_braces}
        & f31 & \texttt{php\_entry1} \\
      f2 & \texttt{variable1} & f17 & \texttt{right\_braces}
        & f32 & \texttt{php\_entry2} \\
      f3 & \texttt{semicolon} & f18 & \texttt{comma}
        & f33 & \texttt{php\_entry3} \\
      f4 & \texttt{int\_num} & f19 & \texttt{string\_literal}
        & f34 & \texttt{comment\_line} \\
      f5 & \texttt{equal} & f20 & \texttt{EOF}
        & f35 & \texttt{comment} \\
      f6 & \texttt{not\_equal} & f21 & \texttt{variable2}
        & f36 & \texttt{comment1} \\
      f7 & \texttt{concatenate} & f22 & \texttt{double\_num}
        & f37 & \texttt{string\_literal} \\
      f8 & \texttt{less} & f23 & \texttt{double\_num\_exp}
        & f38 & \texttt{escape} \\
      f9 & \texttt{greater} & f24 & \texttt{equal2}
        & f39 & \texttt{escape\_x} \\
      f10 & \texttt{plus} & f25 & \texttt{equal3}
        & f40 & \texttt{escape\_x1} \\
      f11 & \texttt{multiply} & f26 & \texttt{not\_equal2}
        & f41 & \texttt{double\_num\_exp\_end} \\
      f12 & \texttt{minus} & f27 & \texttt{not\_equal3}
        & f42 & \texttt{double\_num\_end} \\
      f13 & \texttt{comment} & f28 & \texttt{greater\_equal}
        & s & Počáteční stav \\
      f14 & \texttt{left\_parent} & f29 & \texttt{less\_equal} & & \\
      f15 & \texttt{right\_parent} & f30 & \texttt{php\_entry} & & \\

    \end{tabular}

    \newpage

    \subsection{Pravidla gramatiky}

    \begin{tabular}{ l r l }

      \hline
      \texttt{C} & $\rightarrow$ & \texttt{i} \\
      \hline
      \texttt{C} & $\rightarrow$ & \texttt{C OPERATION C} \\
      \hline
      \texttt{plus} & $\rightarrow$ & \texttt{+} \\
      \texttt{minus} & $\rightarrow$ & \texttt{-} \\
      \texttt{divide} & $\rightarrow$ & \texttt{/} \\
      \texttt{concat} & $\rightarrow$ & \texttt{.} \\
      \texttt{less} & $\rightarrow$ & \texttt{<} \\
      \texttt{lessequal} & $\rightarrow$ & \texttt{<=} \\
      \texttt{greater} & $\rightarrow$ & \texttt{>} \\
      \texttt{greaterequal} & $\rightarrow$ & \texttt{>=} \\
      \texttt{equal} & $\rightarrow$ & \texttt{===} \\
      \texttt{nonequal} & $\rightarrow$ & \texttt{!==} \\
      \hline
      \texttt{C} & $\rightarrow$ & \texttt{(C)} \\
      \hline

    \end{tabular}

    Legenda:

    \begin{tabular}{l l l}
      \texttt{C} & -- & nonterminál \\
      \texttt{i} & -- & terminál -- identifikátor \\
      \texttt{plus, minus, ...} & -- & nontermální reprezentace znaménka \\
      \texttt{OPERATION} & -- & \texttt{\{plus, minus, ...\}}\\
    \end{tabular}

    \subsection{LL gramatika}

    \begin{tabular}{ l r l }

      \hline
      \texttt{PROGRAM} & $\rightarrow$ & \texttt{<?php MAIN} \\
      \texttt{MAIN} & $\rightarrow$ & \texttt{MAIN MAIN} \\
      \texttt{MAIN} & $\rightarrow$ & \texttt{eof} \\
      \texttt{MAIN} & $\rightarrow$ & \texttt{STATEMENTS} \\

      \hline
      \texttt{MAIN} & $\rightarrow$ &
        \texttt{function id ARGDEF \{ STATEMENTS \}} \\
      \texttt{ARGDEF} & $\rightarrow$ & \texttt{variable NEXTARG} \\
      \texttt{ARGDEF} & $\rightarrow$ & $\epsilon$ \\
      \texttt{NEXTARG} & $\rightarrow$ & \texttt{, variable} \\
      \texttt{NEXTARG} & $\rightarrow$ & $\epsilon$ \\

      \hline
      \texttt{STATEMENTS} & $\rightarrow$ & \texttt{return expr ;} \\
      \texttt{STATEMENTS} & $\rightarrow$ &
        \texttt{variable = ASSIGNMENT} \\
      \texttt{STATEMENTS} & $\rightarrow$ &
        \texttt{if (EXPR) \{ STATEMENTS \} else \{ STATEMENTS \}} \\
      \texttt{STATEMENTS} & $\rightarrow$ &
        \texttt{while (EXPR) \{ STATEMENTS \}} \\
      \texttt{STATEMENTS} & $\rightarrow$ & \texttt{$\epsilon$} \\
      \texttt{STATEMENTS} & $\rightarrow$ &
        \texttt{STATEMENTS STATEMENTS} \\

      \hline
      \texttt{ASSIGNMENT} & $\rightarrow$ & \texttt{expr ;} \\
      \texttt{ASSIGNMENT} & $\rightarrow$ & \texttt{id CALL ;} \\

      \hline
      \texttt{CALL} & $\rightarrow$ & \texttt{( PARAMETERS )} \\
      \texttt{PARAMETERS} & $\rightarrow$ & \texttt{PAR NEXTPAR} \\
      \texttt{PARAMETERS} & $\rightarrow$ & \texttt{$\epsilon$} \\
      \texttt{NEXTPAR} & $\rightarrow$ & \texttt{, PAR NEXTPAR} \\
      \texttt{NEXTPAR} & $\rightarrow$ & \texttt{$\epsilon$} \\

      \hline
      \texttt{PAR} & $\rightarrow$ & \texttt{int\_num} \\
      \texttt{PAR} & $\rightarrow$ & \texttt{double\_num} \\
      \texttt{PAR} & $\rightarrow$ & \texttt{string\_literal} \\
      \texttt{PAR} & $\rightarrow$ & \texttt{null} \\
      \texttt{PAR} & $\rightarrow$ & \texttt{false} \\
      \texttt{PAR} & $\rightarrow$ & \texttt{true} \\
      \texttt{PAR} & $\rightarrow$ & \texttt{variable} \\
      \hline

    \end{tabular}

    \newpage

%%%%%%%%%%%%%%%%%%%%%%%%%%%%%%%%%%%%%%%%%%%%%%%%%%%%%%%%%%%%%%%%%%%%%%%%%%%%%%
  \section{Metriky kódu} \label{metriky}
%%%%%%%%%%%%%%%%%%%%%%%%%%%%%%%%%%%%%%%%%%%%%%%%%%%%%%%%%%%%%%%%%%%%%%%%%%%%%%

    \setlength{\parskip}{0cm}
    \paragraph{Počet souborů:} 15 souborů
    \paragraph{Počet řádků zdrojového textu:} 6217 řádků
    \paragraph{Velikost statických dat:} 948B
    \paragraph{Velikost spustitelného souboru:} 71556B (systém Linux, 64
    bitová architektura, při pře\-kladu bez ladicích informací)

%%%%%%%%%%%%%%%%%%%%%%%%%%%%%%%%%%%%%%%%%%%%%%%%%%%%%%%%%%%%%%%%%%%%%%%%%%%%%%

\end{document}
