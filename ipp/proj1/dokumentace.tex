\documentclass[10pt,a4paper,final]{article}
% cestina a fonty
\usepackage[czech]{babel}
\usepackage[utf8]{inputenc}
\usepackage[T1]{fontenc}
\usepackage{lmodern}
\usepackage{textcomp}
\usepackage{times}
% odsazeni prvniho radku
\usepackage{indentfirst}
% balicky pro odkazy
\usepackage[bookmarksopen,colorlinks,plainpages=false,urlcolor=blue,
unicode,linkcolor=black]{hyperref}
\usepackage{url}
% obrazky
\usepackage[dvipdf]{graphicx}
% velikost stranky
\usepackage[top=3.5cm, left=2.5cm, text={17cm, 24cm}, ignorefoot]{geometry}

\begin{document}

%%%%%%%%%%%%%%%%%%%%%%%%%%%%%%%%%%%%%%%%%%%%%%%%%%%%%%%%%%%%%%%%%%%%%%%%%%%%%%%%

  % nastaveni cislovani
  \pagestyle{plain}
  \pagenumbering{arabic}
  \setcounter{page}{1}
  
  % nastaveni mezery mezi odstavci a odsazeni prvniho radku
  \setlength{\parindent}{1cm}
  \setlength{\parskip}{0.5cm plus4mm minus3mm}
  
  \noindent
  Dokumentace úlohy XQR: XML Query v PHP 5 do IPP 2013/2014 \\
  Jméno a příjmení: Roman Blanco \\
  Login: xblanc01 \\
  


%%%%%%%%%%%%%%%%%%%%%%%%%%%%%%%%%%%%%%%%%%%%%%%%%%%%%%%%%%%%%%%%%%%%%%%%%%%%%%%%
  \section{Úvod} \label{uvod}
%%%%%%%%%%%%%%%%%%%%%%%%%%%%%%%%%%%%%%%%%%%%%%%%%%%%%%%%%%%%%%%%%%%%%%%%%%%%%%%%

    Tato dokumentace pojednává o~skriptu napsaném v~jazyce PHP, jehož účelem je
    vyhodnocování dotazu podobného příkazu \texttt{SELECT} jazyka SQL, který je 
    prováděný nad vstupním XML souborem. Níže v~dokumentu naleznete popis
    implementace jednotlivých částí skriptu, jako zpracovávání parametru nebo
    aplikaci dotazu \texttt{SELECT} nad XML zdrojem.

%%%%%%%%%%%%%%%%%%%%%%%%%%%%%%%%%%%%%%%%%%%%%%%%%%%%%%%%%%%%%%%%%%%%%%%%%%%%%%%%
  \section{Zpracování parametrů} \label{zpracovani-parametru}
%%%%%%%%%%%%%%%%%%%%%%%%%%%%%%%%%%%%%%%%%%%%%%%%%%%%%%%%%%%%%%%%%%%%%%%%%%%%%%%%

    Ke~zpracování parametrů zadaných při spuštění skriptu je použita standardní
    funkce jazyka PHP -- \texttt{getopt()}. Funkce usnadňuje načítání
    parametrů tím, že stačí určit názvy přepínačů a o~načtení dat se postará
    funkce sama. Jelikož však \texttt{getopt()} povoluje zápisy, které 
    jsou vzhledem k~zadání nesprávné, je potřeba tyto možné výskyty chybných
    parametrů ošetřit.
    
    Jednou z~problemových situací je zadání dvou stejných přepínačů. Tento
    problem však lze ošetřit porovnáním počtu zadaných parametrů
    s počtem zpracovaných parametrů. Pokud si nejsou rovny, parametry nebyly
    korektně zadány a skript bude ukončen s příslušnou návratovou hodnotou.
    
    Další potíž této funkce nastává při načítání zkrácených přepínačů.
    Tedy například pokud skript pracuje s přepínačem \texttt{-n}, ale bude 
    spuštěn s chybně zapsaným přepínačem obsahující řetězec \texttt{"n"}
    (například \texttt{-ano}), funkce s ním bude i přesto pracovat jako se 
    správně zadaným parametrem \texttt{-n}.
    Tento problem je ve skriptu vyřešen kontrolou názvů přepínačů
    pomocí regulerních výrazů.

%%%%%%%%%%%%%%%%%%%%%%%%%%%%%%%%%%%%%%%%%%%%%%%%%%%%%%%%%%%%%%%%%%%%%%%%%%%%%%%%
  \section{Načtení zdrojového XML souboru}
%%%%%%%%%%%%%%%%%%%%%%%%%%%%%%%%%%%%%%%%%%%%%%%%%%%%%%%%%%%%%%%%%%%%%%%%%%%%%%%%

    Pro zjednodušení práce s~daty uloženými ve~zdrojovém XML souboru je použita
    třída \texttt{SimpleXMLElement}, jejíž metody umožňují veškerou potřebnou
    práci s~XML daty. Třída ze zdrojového XML kódu vytvoří nový stejnojmenný
    objekt, tedy vícerozměrné pole s~atributy podle zdrojového XML kódu, tak
    aby bylo následně možné s daty pracovat pomocí ostatních metod této třídy.

%%%%%%%%%%%%%%%%%%%%%%%%%%%%%%%%%%%%%%%%%%%%%%%%%%%%%%%%%%%%%%%%%%%%%%%%%%%%%%%%
  \section{Zpracování SELECT dotazu} \label{zpracovani-select}
%%%%%%%%%%%%%%%%%%%%%%%%%%%%%%%%%%%%%%%%%%%%%%%%%%%%%%%%%%%%%%%%%%%%%%%%%%%%%%%%

    Data ze~zdrojového souboru jsou tříděná na základě SELECT dotazu, zadaného
    parametrem \texttt{query} nebo zapsaného ve~vstupním souboru s~dotazem, jež
    se načítá parametrem \texttt{qf}.
    Dotaz je rozdělený na podčásti \texttt{SELECT}, \texttt{LIMIT}, 
    \texttt{FROM}, \texttt{WHERE} a \texttt{ORDER BY}. Každá z těchto podčástí
    je načítána pomoci regularních výrazů a ukládana do pole, jehož atributy
    jsou tvořeny názvy daných podčástí.

    Nejsložitější k zachycení je část \texttt{WHERE}, jelikož podle zadání
    může být zapsané libovolné množství výrazů porovnaných logickými výrazy
    \texttt{AND} nebo \texttt{OR}, nebo může být celý výraz negován.
    Z toho důvodu při načítání části \texttt{WHERE} namísto, aby se regularním
    výrazem zachytila jen potřebná data, zachytí se kompletní výraz, jež je
    dále rozdělován a porovnáván, zda-li vyhovuje syntaxi dané zadáním projektu.

%%%%%%%%%%%%%%%%%%%%%%%%%%%%%%%%%%%%%%%%%%%%%%%%%%%%%%%%%%%%%%%%%%%%%%%%%%%%%%%%
  \section{Vyhodnocení dotazu nad daty} \label{vyhodnoceni-dotazu}
%%%%%%%%%%%%%%%%%%%%%%%%%%%%%%%%%%%%%%%%%%%%%%%%%%%%%%%%%%%%%%%%%%%%%%%%%%%%%%%%

    Pro výběr dat požadovaných na základě \texttt{SELECT} dotazu je použita 
    metoda \texttt{xpath()}. Ta umožnuje pomocí jazyka xpath získat potřebná
    data z~pole dat načtených z~XML souboru.
    
    Jako první je načítaná část dotazu \texttt{FROM}, jelikož dle zadání
    jsou vždy vyhledávána data objevující se v~prvním výskytu vyhovujícímu
    výrazu \texttt{FROM}. Poté jsou části vyhodnocovány v pořadí
    \texttt{SELECT}, \texttt{WHERE} a na konec je výsledek případně zkrácen 
    na~stanovený limit, tedy požadavkem z~části \texttt{LIMIT}.
    
%%%%%%%%%%%%%%%%%%%%%%%%%%%%%%%%%%%%%%%%%%%%%%%%%%%%%%%%%%%%%%%%%%%%%%%%%%%%%%%%
    
\end{document}
