\documentclass[10pt,a4paper,final]{article}
% cestina a fonty
\usepackage[czech]{babel}
\usepackage[utf8]{inputenc}
\usepackage[T1]{fontenc}
\usepackage{lmodern}
\usepackage{textcomp}
\usepackage{times}
% odsazeni prvniho radku
\usepackage{indentfirst}
% balicky pro odkazy
\usepackage[bookmarksopen,colorlinks,plainpages=false,urlcolor=blue,
unicode,linkcolor=black]{hyperref}
\usepackage{url}
% obrazky
\usepackage[dvipdf]{graphicx}
% velikost stranky
\usepackage[top=3.5cm, left=2.5cm, text={17cm, 24cm}, ignorefoot]{geometry}

\begin{document}

%%%%%%%%%%%%%%%%%%%%%%%%%%%%%%%%%%%%%%%%%%%%%%%%%%%%%%%%%%%%%%%%%%%%%%%%%%%%%%%

  % nastaveni cislovani
  \pagestyle{plain}
  \pagenumbering{arabic}
  \setcounter{page}{1}
  
  % nastaveni mezery mezi odstavci a odsazeni prvniho radku
  \setlength{\parindent}{1cm}
  \setlength{\parskip}{0.5cm plus4mm minus3mm}
  
  \noindent
  Dokumentace úlohy CST: C Stats v~Python 3 do IPP 2013/2014 \\
  Jméno a příjmení: Roman Blanco \\
  Login: xblanc01 \\

%%%%%%%%%%%%%%%%%%%%%%%%%%%%%%%%%%%%%%%%%%%%%%%%%%%%%%%%%%%%%%%%%%%%%%%%%%%%%%%
  \section{Úvod} \label{uvod}
%%%%%%%%%%%%%%%%%%%%%%%%%%%%%%%%%%%%%%%%%%%%%%%%%%%%%%%%%%%%%%%%%%%%%%%%%%%%%%%

    Tato dokumentace pojednává o~skriptu napsaném v~jazyce Python.
    Jeho účelem je analyzovat soubory obsahující zdrojový kód jazyka C podle
    standardu ISO C99.
    Typ prováděné analýzy je vždy daný parametrem při spuštění skriptu. Níže
    v~dokumentu naleznete popis implementace těchto typů analýzy či dalších
    částí skriptu.

%%%%%%%%%%%%%%%%%%%%%%%%%%%%%%%%%%%%%%%%%%%%%%%%%%%%%%%%%%%%%%%%%%%%%%%%%%%%%%%
  \section{Zpracování parametrů} \label{zpracovani-parametru}
%%%%%%%%%%%%%%%%%%%%%%%%%%%%%%%%%%%%%%%%%%%%%%%%%%%%%%%%%%%%%%%%%%%%%%%%%%%%%%%

    Ke zpracování parametrů skript využívá standardní knihovnu jazyka Python --
    \texttt{argparse}, díky které stačí nastavit ve skriptu názvy přepínačů a
    typ ukládané hodnoty.

    Použitím této knihovny je zároveň zajištěno částečné ověření parametrů.
    Například pokud je skript spuštěný s~parametrem, který není mezi
    nastavenými, skript se automaticky ukončí. Jelikož však výchozí metoda
    obstarávající vypisovaní chyby parametrů nevyhovuje tím, že při chybě
    parametrů vrací hodnotu 2, je potřeba předefinovat tu metodu vlastní.

    Co však tato knihovna neřeší, je vícenásobné zadání stejného parametru.
    Tento problém je vyřešený sečtením parametrů a ověřením, zda je daný
    parametr zadán více než jedenkrát. Stejným způsobem je vyřešeno i zadání
    více parametrů, které jsou dle zadání nekombinovatelné.

%%%%%%%%%%%%%%%%%%%%%%%%%%%%%%%%%%%%%%%%%%%%%%%%%%%%%%%%%%%%%%%%%%%%%%%%%%%%%%%
  \section{Analýza souboru} \label{analyza}
%%%%%%%%%%%%%%%%%%%%%%%%%%%%%%%%%%%%%%%%%%%%%%%%%%%%%%%%%%%%%%%%%%%%%%%%%%%%%%%

    \subsection{Klíčová slova} \label{klicova-slova}

      Účelem této analýzy je sečíst všechna klíčová slova ze zdrojového
      souboru, která se nevyskytují v~poznámkách, makrech a řetězcích.
      Prvním krokem je tedy odstranění částí, které nemají být prohledávány,
      z~načteného obsahu souboru. To je řešeno pomocí regulárních výrazů,
      kterými se vyhledají a odstraní všechny komentáře, řetězce a makra. Poté
      jsou, opět pomocí regulárních výrazu, vyhledána všechna slova, u~kterých
      se zjišťuje, za jsou jedním z~hledaných klíčových slov. Pro tento
      účel je do skriptu importován modul \texttt{ckeyword.py}, v~němž je
      vytvořený seznam klíčových slov jazyka C, s~nimiž se vyhledaná slova 
      porovnávají.

    \subsection{Jednoduché operátory} \label{operatory}

      Při analýze operátorů je potřeba, podobně jako u~analýzy klíčových slov,
      odstranit všechny komentáře, řetězce a makra, a navíc ještě vyhledat a
      odstranit deklarace ukazatelů, jelikož v~tomto případě není \texttt{*}
      považována za operátor. Poté jsou již operátory vyhledány regulárním
      výrazem a sečteny.

    \subsection{Identifikátory} \label{identifikatory}

      Implementace analýzy identifikátorů je téměř totožná jako analýza
      klíčových slov. Jediným rozdílem je, že se sečtou slova, která nevyhovují
      žádnému ze slov uložených v~seznamu v~importovaném modulu
      \texttt{ckeyword.py}.

    \subsection{Textový řetězec \texttt{pattern}} \label{retezec}

      Analýza textového řetězce se spouští zadáním parametru
      \texttt{-w=pattern}, kde \texttt{pattern} je přesný řetězec, který má být
      v~analyzovaném souboru vyhledán. Počet výskytů se v~této analýze zjišťuje
      metodou \texttt{count}.

    \subsection{Komentáře} \label{komentare}

      Počítání komentářů je implementováno jako konečný automat. Obsah
      zdrojového souboru se prochází znak po znaku, přičemž na základě
      posledních dvou znaků se nastavují hodnoty booleovských proměnných, které
      následně reprezentují stavy konečného automatu -- proměnná
      \texttt{comment} informuje, že aktuálně načítané znaky jsou součástí
      komentáře, \texttt{oneLineComment} zda se jedná o~jednořádkový komentář a
      proměnná \texttt{string} informuje že aktuální znaky jsou součástí
      řetězce.

      Řetězce jsou v~konečném automatu rozpoznávány, jelikož je nelze odstranit
      před samotným počítáním znaků (mohou být součástí komentáře), avšak znaky
      označující začátek či konec komentáře mohou být zapsané v~řetězci.
      V~takovémto případě se však nejedná o~komentář a tyto znaky se nemají
      započítávat.

      Tato implementace načítání komentářů díky své konstrukci splňuje bonusové
      rozšíření, spočívající v~započítávání znaků komentáře i v~rámci definic
      makra.

%%%%%%%%%%%%%%%%%%%%%%%%%%%%%%%%%%%%%%%%%%%%%%%%%%%%%%%%%%%%%%%%%%%%%%%%%%%%%%%
  \section{Výstup} \label{vystup}
%%%%%%%%%%%%%%%%%%%%%%%%%%%%%%%%%%%%%%%%%%%%%%%%%%%%%%%%%%%%%%%%%%%%%%%%%%%%%%%

    Po analýze jednoho či více souborů je potřeba zpracovaná data vypsat podle 
    volby buďto na standardní výstup, nebo do souboru. Jelikož je však ve 
    výstupu součet výskytu zarovnaný doprava, je potřeba zjistit odsazení pro 
    jednotlivé soubory. Pro odsazení je potřeba nejdříve nalézt nejdelší
    řetězcovou délku cesty a součtu výskytu hledaných prvků u~analyzovaných
    souborů. Od této hodnoty se poté odečítá řetězcová délka cesty a počtu
    výskytů pro každý vypisovaný soubor.

%%%%%%%%%%%%%%%%%%%%%%%%%%%%%%%%%%%%%%%%%%%%%%%%%%%%%%%%%%%%%%%%%%%%%%%%%%%%%%%
    
\end{document}
